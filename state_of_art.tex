In this chapter a survey of pervasive healthcare applications is presented. In order to gain a greater understanding of which are the building blocks of a typical \acf{IoT} system, a reference model is also presented.

% Aqui é necessário algum contexto ao leitor. Uma vez que vais entregar um estado da arte, é dificil desacoplares isto completamente do teu trabalho.
% Seria importante uma frase introdutória do estilo:
% "Being central to the work described in this dissertation/report/document, in this chapter..."

\todo[inline]{To-do: Redo introductory paragraph (see coments in latex).}

\section{Internet of Things}

\subsection{Fundamentals of \acs{IoT}}

% Por ser tão central ao trabalho, colocar a definição dentro de uma caixa para ficar mais realçada pode fazer sentido. Avalia isso.

\acl{IoT} (or \acs{IoT}) is an emerging communication paradigm, often hailed as the driver of the Fourth Industrial Revolution \cite{Aceto2020}. \bigskip

\todo[inline]{To-do: Highlight definition using a box (see coments in latex).}

The definition of this concept has evolved over time with the development of other technologies such as data analytics, embedded systems, sensors, etc. Fundamentally, it describes a strategy supported on the development of networks of smart devices that exchange and process information through Machine-to-Machine (M2M) communications, usually based on the \acf{IP} \cite{Baker2017}. This technology enables ubiquitous systems to gather remarkable amounts of information regarding the surrounding environment, which can later be turned into insight through the usage of data fusion and data analytics tools, like Machine Learning. \bigskip

In the specific context of healthcare, this technology can provide many benefits as it enables remote and continuous health monitoring \cite{Doukas2012, Wu2020, Fan2014}. It allows non-critical patients to be monitored from the comfort of their own houses, rather than in hospitals or clinics, reducing the strain on scarce hospital resources such as health professionals or beds. This is particularly beneficial to those who live in rural areas, with limited access to healthcare services. It enables elderly people and those with chronic diseases to have greater control over their own health, thus allowing them to live more independently. Moreover, with the automatization of medical procedures, these systems can make healthcare infrastructures more efficient and therefore lower the costs of healthcare \cite{Catarinucci2015, Adame2018}. Particularly, in the realm of clinical research, by analyzing the data collected by these ubiquitous systems, it may be possible to find new relationships between certain pathologies and different physiological signals, such as variations in body temperature or heart rate \cite{Choi2016}. These correlations, commonly referred to as biomarkers, can be used by these systems to assist clinical decisions, enabling novel predictive, prognostic, and diagnostic processes in healthcare.

\todo[inline]{To-do: Review paragraph — does it emphasizes too much remote health monitoring? should be more focus on demonstrating value for hospitals or is it fine as it is?}

\section{A Reference Model for Pervasive Healthcare Applications}

\todo[inline]{To-do: Review this paragraph and section later.}

Reference models provides an abstract framework for designing systems and a set of commonly recommended practices for the application domain. It serves as a starting point in the design process, enabling the comprehension of complex systems by breaking them down into simple and distinct functional layers, while also defining some common terminology used in its domain. \bigskip

In 2014, the \acs{IoT} World Forum (IoTWF) architectural committee published an \acs{IoT} architectural reference model, composed by seven layers as shown in Figure \ref{fig:iotwf-referencemodel}. This model \cite{Cisco2014} provides a simple and clean functional view into the different components of an \acs{IoT} system, without narrowing the scope or locality of its components. However, from a hardware perspective, in this work we will focus on the most common approach taken by researchers, using 3 distinct components: 

\begin{itemize}
    \item \textbf{Endpoint} or \textbf{edge} nodes (corresponding to Layer 1), which interact with the physical world, capturing data.
    \item \textbf{Gateway} devices (Layer 3), which connect to one or multiple \textbf{edge} nodes, filtering and aggregating the data generated by these and communicating it to a central server; 
    \item \textbf{Central} server (Layers 4-6), which is responsible for collecting, storing and analyzing the captured data in order to provide users with valuable insight;
\end{itemize}

While this model can be used to generically develop \acs{IoT} systems for any industry (\textit{e.g.} from agriculture to smart cities), in the context of the dissertation we will focus on pervasive healthcare applications. \bigskip

\todo[inline]{To-do: Make new image based on this one. (!!)}

\begin{figure}[H]
    \centering
    \includegraphics[width=0.85\linewidth]{images/iotwf-referencemodel.png}
    \caption[IoT reference model published by IoTWF.]{IoT reference model published by IoTWF. Source: \cite{Cisco2014}.}
    \label{fig:iotwf-referencemodel}
\end{figure}

\subsection{Layer 1: Physical Devices and Controllers}
\label{sec:iot-model-layer1}

The first layer of the model \cite{Cisco2014} is the physical devices and controller layer. This layer houses the ``things in the \acl{IoT}: the endpoint devices composed of sensors and actuators that perceive and interact with the physical world. Through those interactions, the devices generate data, which is then sent across the network for analysis and storage. \bigskip

Wearable, wireless, and non-intrusive devices are viewed as one of the key components of \acs{IoT}-based healthcare systems \cite{Baker2017}. In recent years there has been remarkable progress on the development of wearable devices, driven by recent technological breakthroughs in the miniaturization of sensors and microfabrication processes \cite{Adame2018, Catarinucci2015}. These devices allow patients to be monitored while retaining their mobility, thus increasing the comfort of these users. The drawback of this approach lies on the restrictions imposed on the devices. Due to the nature of this technology, most of these units require a portable energy source, a battery, which implies reduced memory, computation, and connectivity capabilities in order to minimize energy consumption and maximize their lifetime. Shorter lifetimes translate into higher maintenance costs, as these devices need to be replaced more often. \bigskip

Another point to consider is the data requirements of the system, namely how much data is generated and which type of data is transmitted by the devices. Some applications can include a single temperature sensor or heart rate sensor, while more complex systems can include pulse oximetry, electrocardiogram (\acs{ECG}), respiration rate sensors, etc. \cite{Wu2020}. From the literature \cite{Wu2020, Wu2019, Adame2018, MinhDang2019}, we can also classify the sensors used in these devices into three distinct categories based on the signals that can be extracted from them, as shown in Table \ref{tab:layer1-sensors}:

\begin{itemize}
    \item \textbf{Physiological Sensors}: used for evaluating the patients health condition.
    \item \textbf{Activity / Motion sensors}: used for detecting fall events, determining the patients location and travelled distance, estimating the patients body posture, etc.
    \item \textbf{Environmental Sensors}: used for assessing environment conditions and possible hazards, \textit{e.g.} gas leaks in a patients home or an industrial workplace \cite{Wu2019}.
\end{itemize}

\renewcommand{\arraystretch}{2}
\begin{table}[H]
    \centering
    \begin{tabular}{l|l}
        \textbf{Sensor Categories} & \textbf{Examples} \\ 
        \hline
        Vital Signs Monitoring & \makecell{Blood Pressure, \acs{ECG}, \acs{PPG}, Body Temperature, \\ Respiratory Rate, Galvanic Skin Response, \\ Pulse Oximetry, Glucose Level Sensors} \\
        Activity Monitoring & Accelerometer, Gyroscope, Magnetometer\\
        Environmental Monitoring & Air Temperature, Barometer, Humidity, Gas Sensors \\
    \end{tabular}
    \caption[Type of sensors commonly used in pervasive healthcare applications.]{Type of sensors commonly used in pervasive healthcare applications. Adapted from \cite{MinhDang2019}.}
    \label{tab:layer1-sensors} 
\end{table}
\renewcommand{\arraystretch}{1}

\subsection{Layer 2: Connectivity}
\label{sec:iot-model-layer2}

The second layer of the model focuses on connectivity, on linking the different components of our system, ensuring reliable and timely data transmissions. This includes all communications within the system, which can be divided into two categories: communications within the local network (\textit{e.g.} between edge nodes and the gateway devices), and communications between the edge of the local network (\textit{e.g.} gateway devices) and the central server. \bigskip

\subsubsection{Communication Protocols}

Technology is designed with particular use cases in mind. They catalyze development and each different technology has its own advantages and disadvantages, according to its use. For instance, short range wireless protocols are limited by transmission range, but long range protocols have a higher energy consumption, which may be unviable for networks with very constrained devices. Each protocol defines their own frame formats and communicates within certain frequency bands, some of which may require licenses. Using licensed frequency bands can provide a better performance as it ensures greater reliability since the network operator grants you exclusivity of frequency spectrum within a certain area. \bigskip

From the literature, we can identify a set of key requirements that drive the decision of the communication protocols \cite{Baker2017, Catarinucci2015, Adame2018}:

\begin{itemize}
    \item \textbf{Energy consumption}: For networks composed of energy constrained devices, the communication protocol should be lightweight and energy efficient in order to maximize the devices lifetime. 
    \item \textbf{Latency}: Certain applications deal with time critical events, for example the detection of health emergencies \cite{Catarinucci2015}. In these cases, any delays in the communications can cause great detriment to the patients well-being, making it crucial to minimize them.
    \item \textbf{Reliability}: Depending on the critical nature of the data that is being communicated, the network stack may need to implement processes such as error-detection, retransmission or handshakes in the communications to ensure more robust transmissions, \textit{e.g.} as implemented in TCP/IP based protocols. Generally, these features come at the cost of greater latency. Therefore, when choosing the communication protocol, a balance must be found between reliability and latency.
    \item \textbf{Security}: Security is one of the most important requirements of any system, but this is especially true for healthcare applications. Due to the sensitive nature of the information, it is crucial to secure it from malicious actors. Communication protocols must implement security mechanisms, such as encryption or data integrity verifications, that ensure the transmissions are not compromised in transit, thus denying third parties the ability to snoop or tamper the transmissions. This issue is studied in depth in \cite{Gope2016}.
    \item \textbf{Interoperability}: To ensure the interoperability of intrinsically different modules of the system it is imperative to choose protocols that are widely accepted and supported in the application domain. This also contributes to the longevity and maintenance of the system, as these will most likely remain supported for longer time periods. 
    \item \textbf{Range}: The communication protocol must ensure the devices can communicate within the required transmission range.
    \item \textbf{Scalability}: These systems contain an enormous amount of devices, which must be uniquely identified. The communication protocol must ensure that every device in the network is addressable, and that performance is not severely impacted by the addition of new devices.
    \item \textbf{Throughput}: The communication protocol should ensure there is enough bandwidth to handle all communications within the designated transmission range. Even within similar technologies, this can vary wildly with the range, as seen in Figure \ref{fig:communication-protocols-throughput}.  

\end{itemize}

\begin{figure}[H] 
    \centering
    \includegraphics[width=0.55\linewidth]{images/communication-protocols-throughput.png}
    \caption{Throughput versus Transmission range for four communications protocols (quais?) with similar ranges. Source: \cite{10.5555/3161403}}
    \label{fig:communication-protocols-throughput}
\end{figure}

% IEEE 802.15.4?
Regarding communications within the local network, these generally have short transmission ranges \cite{Baker2017}. Wearable devices can be often arranged in networks, aptly designated ``Wireless Body Area Network. The most widely adopted protocol is \acf{BLE}, a low-energy version of the classic Bluetooth protocol \cite{Doukas2012, Wu2019, Wu2020}. ZigBee and \acf{RFID} are also used in many systems in this domain, particularly in more asset tracking oriented applications \cite{Fuhrer2006, Catarinucci2015, Adame2018}. Despite the absence of a universal specification for \acs{RFID}, the most widely used standard is the EPCglobal Gen2 RFID \cite{EPCglobal2006}, which is the standard considered in this work. discussed in this work. For the sake of simplicity, we use \acs{EPC/RFID}  will be used to designate it. \bigskip

Out of the abovementioned technologies, \acs{BLE} offers greater throughput, better security, and nearly the lowest energy consumption \cite{dementyev2013power}. Table \ref{tab:comparsion-shortrangeprotocols} shows a comparison between the different protocols.

\renewcommand{\arraystretch}{1.5}
 \begin{table}[H]
     \centering
     \begin{tabular}{r|l|l|l}
         & \textbf{\acl{BLE}} & \textbf{EPC/\acs{RFID}}& \textbf{ZigBee}  \\ \hline
         \textbf{Band of operation} & 2.4 GHz & LF, HF, UHF, EHF & 2.4 GHz \\
         \textbf{Communication} & Bidirectional & \makecell{Unidirectional \\ (Bidirectional for\\ Active tags)} & Bidirectional \\
         \textbf{Topology} & \makecell{ Point-to-Point, Piconet, \\ Broadcast, Mesh } & Point-to-Point & Mesh \\
         \textbf{Range} & <100m & \makecell{<10m, (100m for\\ Active tags)} & 20m \\
         \textbf{Data rate} & 1Mb/s (up to 2Mb/s) & 40kb/s & 250kb/s \\
         \textbf{\acs{IP} Stack} & \xmark & \xmark & \cmark \\
         \textbf{Security Features} & \makecell{AES-128, Secure pairing \\ prior to key exchange} & \xmark & \makecell{AES-128 (Optional),\\ Network key shared \\across network, \\ Optional link key to \\ secure application layer \\ communications}\\
     \end{tabular}
     \caption[Comparison between the more common communication protocols used within short range]{Comparison between the more common communication protocols used within short range. Adapted from \cite{Baker2017}.}
     \label{tab:comparsion-shortrangeprotocols}
 \end{table} 
\renewcommand{\arraystretch}{1}

Regarding communications between the gateway devices and the central server, most researchers try to make use of existing infrastructure in order to facilitate the deployment of new systems. This means, that the communications between the gateway devices and the central server are often done through generic IP-based protocols such as Wi-Fi and Ethernet \cite{Adame2018, Fuhrer2006, Wu2020, Catarinucci2015}. \bigskip

% In \cite{Wu2020} the authors developed a smartphone app (Android OS) that connects to the wearable devices through \acs{BLE}, allowing any smartphone to serve as a gateway device...

\subsubsection{Application Protocols}
\todo[inline]{To-do: Review this section later.}

So far we have discussed the underlying networking technologies that link the devices in our system. But according to the OSI Model, many of these technologies do not define the application layer: how the devices communicate with each other, how the data is formatted, if there is a hierarchy within the network, etc. When considering networks composed of constrained devices, generic web-based protocols such as Hypertext Transfer Protocol (HTTP) may not be adequate for \acs{IoT} applications, which prompts the development of novel lightweight messaging protocols suited for these systems. The most used application layer protocols in \acs{IoT} systems are \acf{MQTT} and \acf{CoAP}. Table \ref{tab:comparsion-applicationprotocols} overviews these widely used protocols. \bigskip

\renewcommand{\arraystretch}{1.5}
\begin{table}[H]
    \centering
    \begin{tabular}{r|l|l}
        %\textbf{Features} 
        & \textbf{\acs{MQTT}}& \textbf{\acs{CoAP}}  \\ \hline
        \textbf{Transport protocol} & TCP/IP & UDP/IP \\
        \textbf{Messaging pattern} & P-S (asynchronous) & R-R (synchronous) \\
        \textbf{Communication model} & Many-to-many & One-to-one \\
        \textbf{Security} & SSL/TLS (Optional) & DTLS \\
        \textbf{Strengths} & \makecell{TCP and Quality of Service (QoS),\\ robust \\communications, easier to implement } & \makecell{Better for lossy networks,\\ lower latency} \\
        \textbf{Weaknesses} & \makecell{Higher overhead and energy\\ consumption than \acs{CoAP}} & \makecell{Not as reliable and \\less supported than MQTT} \\
    \end{tabular}
    \caption[Comparison between \acs{CoAP} and \acs{MQTT} protocols.]{Comparison between \acs{CoAP} and \acs{MQTT} protocols. Adapted from \cite{10.5555/3161403}.}
    \label{tab:comparsion-applicationprotocols}
\end{table} 
\renewcommand{\arraystretch}{1}

In \cite{Rubi2019}, the authors compare these two protocols in greater length, along with the more commonly used web-based protocol Hypertext Transfer Protocol (HTTP). They analyze the latency in the communications (from the edge devices to remote servers) and the RAM usage in the devices for each protocol and for different data sources (respiration rate, oxygen saturation and heart rate signals) and found that \acs{CoAP} presented the best overall performance. Nonetheless, all protocols had very low latencies (less than 1.5s) and low memory usage. \bigskip

The authors indicate that \acs{MQTT} might be more suitable when considering a certain messaging pattern - the ``Publish/Subscribe (P-S) model. In this model, the devices that send messages, called ``publishers, communicate them to an intermediary message broker, through a ``message topic (also called logical channels). The devices that wish to receive messages, called ``subscribers, can subscribe to these topics by requesting it to the message broker. Whenever a publisher sends a message, the broker broadcasts it to all devices that have subscribed to the selected topic. \bigskip

\acs{CoAP} uses a different messaging pattern, called ``Request-Response. In this paradigm, a device sends a request to receive certain data and the second responds to this request. We can observe that the former model is event-driven while the latter is query-based \cite{Rubi2019}. % Due to the nature of how the data is generated in these types of pervasive applications, an event-driven protocol like \acs{MQTT} might be more adequate.

\subsection{Layer 3: Edge (Fog) Computing}
\label{sec:iot-model-layer3}

\acs{IoT} systems may have hundreds or even thousands of sensors generating data multiple times per second, 24 hours per day, which may require an unsustainable amount of network and computing resources. Moreover, certain applications are time critical, where delays in communication can be very detrimental. To minimize these effects, it is crucial to initiate data processing as close to the edge of the network as possible. This paradigm is usually referred to as edge computing, when the data processing occurs at the endpoint devices, or fog computing, when it happens at the edge of local network, \textit{e.g.} in gateway devices. \bigskip

The third layer of the model defines how the system prepares the data for storage and higher level processing for the next layers. However, the endpoint or gateway devices often have limited computing capabilities, so the data processing is generally focused on preprocessing the data in real-time and handling more time critical events. More demanding and thorough data analysis is usually left to the central server. \bigskip

The different processes applied at this stage can be summarized into four distinct categories:

\begin{itemize}
    \item \textbf{Filtering}: Assessing if the data should be processed at a higher level. 
    \item \textbf{Formatting}: Reformatting data to ensure consistent formats for higher-level processing.
    \item \textbf{Cleaning}: Reducing data to minimize the impact of data on the network and higher level processing systems.
    \item \textbf{Evaluation}: Determining whether data represents a threshold or alert. This is especially relevant for applications that deal with time critical events as seen in the previous section.
\end{itemize}

% \todo[inline]{To-do: Add 2 examples of edge computing processes discussed in the systems?} 

\subsection{Layer 4: Data Accumulation}
\label{sec:iot-model-layer4}

The data that is generated by the edge devices is propagated through the system, and eventually reaches the central server. Up to this point, we can state that the systems architecture is event driven. However, most applications cannot make use of the data at the rate it is generated \cite{10.5555/3161403}. In the Data Accumulation layer, we need to define how the system captures the data and stores it, so it becomes usable for applications when needed, thus transiting from event to query-based processing. \bigskip

As the devices continuously generate data, the system will require more and more resources in order to process and store all of this information, raising some concerns regarding how the data can be managed. In \cite{Doukas2012}, the authors propose the usage of cloud platforms as a solution to these problems. This is made possible due to the elasticity in allocating, swiftly and inexpensively, computing and storage resources on-demand, adjusting itself to the needs of each application. We can find 3 distinct types of cloud services: %, Figure \ref{fig:differences-between-cloud-services}:

\begin{itemize}
    \item \textbf{Infrastructure as Service (IaaS)}: Provides control over the remote machine (composed of virtual or dedicated hardware), operative system and middleware. This approach gives system designers the highest level of flexibility over the infrastructure, but requires more maintenance.
    \item \textbf{Platform as a Service (PaaS)}: Provides a simple framework for developing applications, where the service provider manages the underlying infrastructure issues such as software updates and hardware maintenance. 
    \item \textbf{Software as a Service (SaaS)}: Provides the finished applications to be used by the end users, in this case health workers, that enable them to work. A simple example is a web-based email service, such as Gmail or Microsoft Outlook. 
\end{itemize}

\begin{figure}[H]
    \centering
    \includegraphics[width=\linewidth]{images/cloud-services.png}
    \caption[Differences between the cloud offerings and on-premise solutions.]{ Differences between the cloud offerings and on-premise solutions. Source: \cite{RedHat2021}}
    \label{fig:differences-between-cloud-services}
\end{figure}

% \todo[inline]{To-do: Should homomorphic encryption methods be discussed? Discuss what is Hadoop? What are the 5 Vs of Big Data?}

Nonetheless, security and privacy remain as key concerns for the implementation of cloud-based solutions. The information must remain accessible to authorized parties such as healthcare providers, but the patients health data has to be kept private. To solve this, there are two commonly adopted features in the literature: access control policies and data encryption \cite{Baker2017}. Access control policies define who can access the data, by authenticating them (validating the identity of the user) and by authorizing them (ensuring that the user has permissions to perform a given operation). Data encryption ensures that, even if the data is leaked, it is still unreadable to third parties, and therefore sensitive information remains secure and private. \bigskip

Regarding storage solutions, most research works use traditional relational databases (SQL) as the means of storing data \cite{Fuhrer2006, Wu2020, Catarinucci2015, Adame2018}. However, as the data is very heterogenous and unstructured, the authors of \cite{Subahi2019} propose using non-relational databases (NoSQL) which can offer much greater flexibility as the schema in these databases is dynamic. Moreover, NoSQL type data stores outperform SQL databases as the data increases in volume \cite{Xu2014}. Table \ref{tab:comparsion-databasetech} illustrates the differences between these two technologies. 

\renewcommand{\arraystretch}{2}
\begin{table}[H]
    \centering
    \begin{tabular}{r|l|l}
        %\textbf{Features} 
        & \textbf{SQL}& \textbf{NoSQL}  \\ \hline
        \textbf{Type of database} & Relational & Non-relational \\
        \textbf{Database model} & Table-based database & \makecell{Document-based databases, Key-value stores, \\ graph stores, wide column stores} \\
        \textbf{Data type} & Appropriate for structured data & Appropriate for unstructured or semi-structured data \\
        \textbf{Schema} & Strict schema & Dynamic schema \\
        \textbf{Query} & \makecell{Uses Standard Query \\Language (SQL), appropriate for \\ complex query operations} & \makecell{Uses Unstructured Query \\ Languages (\textit{e.g.} UnQL), does not support \\ complex query operations}  \\ 
        \textbf{Scalability} & Vertical & Horizontal \\
        \textbf{Performance} & Lower than NoSQL & Optimized for large datasets \\
    \end{tabular}
    \caption{Comparison between SQL and NoSQL technologies.}
    \label{tab:comparsion-databasetech}
\end{table} 
\renewcommand{\arraystretch}{1}

\subsection{Layer 5: Data Abstraction}
\label{sec:iot-model-layer5}

\todo[inline]{To-do: Complete section with some FHIR development.} 

In the previous layer, we have defined how the system captures the information. In some cases, the collection of data may require the development of multiple concurrent storage solutions, each using different technologies, resulting in a very complex environment. The purpose of this layer is to develop services that simplify how the applications access the data, to reconcile the different data stores and ensure the information is complete and consistent \cite{Cisco2014}. Applications can then interact with these databases through interfaces exposed by these services, designated \acl{API}s (\acs{API}s). \bigskip 

An \acs{API} is a computing interface that defines a set of rules that ``explain how computers and applications communicate with one another, acting as an intermediary between these different components \cite{IBMAPI}. It defines what operations can be performed, how to request them, which are the accepted data types, etc. In this case, it decouples applications from the storage solutions, by encapsulating their functionality behind the interface. This ensures the modularity of the system as the applications become independent of whichever technologies are used in the data stores. \bigskip

Understanding what and how information is shared within the healthcare domain is fundamental. As patients continuously move around the healthcare ecosystem, their health information must be available, discoverable and understandable to different entities (hospitals, laboratories, pharmacies, etc.). This prompts the digitization of medical files and the development of standards for exchanging these records instantly and securely to authorized parties \cite{HL72019}, which are called \acl{EHR}s (\acs{EHR}s). \acs{EHR}s are the digital equivalent of a patients paper-chart, they contain the patients full medical history: previous diagnoses, treatment plans, test results, known allergies, among other details. \bigskip

% Comparison between OpenEHR and FHIR?
One of the most prominent standards for exchanging \acs{EHR}s is \acf{FHIR}. \acs{FHIR} is a standard developed by Health Level Seven International (HL7), which is a non-profit organization involved in the development of international healthcare informatics for over 20 years. \acs{FHIR} builds upon previous data format standards like HL7 v2 and HL7 v3, and is becoming widely adopted within the healthcare industry \cite{Peng2019}. 

% \todo[inline]{To-do: If time allows it, add 2 examples of articles where APIs implemented.} 

\subsection{Layer 6: Application}
\label{sec:iot-model-layer6}

\todo[inline]{To-do: Complete section!}

% Interprets data using software application. Applications may monitor, control and provide reports based on the analysis of the data.
% If the 5 Vs of Big Data are mentioned in the previous section, this is where the VALUE of data is extracted.

The sixth layer is the application layer, where the system ingests the captured data and proceeds to analyze it. Users can then interact with the system through Graphical User Interfaces (UI), which provide different functionalities depending on the application. Some may show simple reports regarding the collected data \cite{Doukas2012, Wu2020}, other may allow users to monitor and control the different components of the system. Thus, in this layer, the system delivers the value for the end users.

% As previously mentioned, one of the greatest benefits \acs{IoT} systems can provide is big data analytics.  

% In healthcare, and particularly in hospitals, clinical professionals use \acl{HIS}s (\acs{HIS}). These systems are used to manage all different aspects of a hospitals operation, medical, administrative, financial and legal. 



% Finally, machine learning to perform diagnostics or provide treatment plans would be extremely valuable in a healthcare context, so a cloud storage framework for healthcare would need to enable value. As all of the characteristics of big data are important to healthcare applications, recent research in this area has focused on storing a wide variety of data generated by voluminous IoT systems in an organized manner that may be useful for later data analysis.

% Centralized and hassle-free collection of data from humans, is a very desired topic in digital health, since it would allow discovery of new digital biomarkers. That is, by acquisition and analysis of electrical/auditory/other physical events of the body, one can find new relationships between certain conditions of the patients and these events [76]. As an example, it is known that dementia has effects on regulation of the body temperature. Therefore, would it be possible to use continuous monitoring of the body temperature and AI to discover Alzheimer development? Although this is not the objective of this project, the proposal makes an important step towards providing large and diverse data to data scientists for analysis. Within this project, we will demonstrate a preliminary example of this ambitious objective, by analyzing 24 hours of data from 3 volunteer who are using a specific drug (i.e. paracetamol or similar), and demonstrate the effect of the drug intake on various parameters, including temperature, emotions, heart rate, blood oxygen, and respiration.

% Level 6 is the application level, where information interpretation occurs. Software at this level interacts with Level 5 and data at rest, so it does not have to operate at network speeds.
% The IoT Reference Model does not strictly define an application. Applications vary based on vertical markets, the nature of device data, and business needs. For example, some applications will focus on monitoring device data. Some will focus on controlling devices. Some will combine device and non-device data. Monitoring and control applications represent many different application models, programming patterns, and software stacks, leading to discussions of operating systems, mobility, application servers, hypervisors, multi-threading, multi-tenancy, etc. These topics are beyond the scope of the IoT Reference Model discussion. Suffice it to say that application complexity will vary widely.
% Examples include:
% ● Mission-critical business applications, such as generalized ERP or specialized industry solutions
% ● Mobile applications that handle simple interactions
% ● Business intelligence reports, where the application is the BI server
% ● Analytic applications that interpret data for business decisions
% ● System management/control center applications that control the IoT system itself and don’t act on the data produced by it.

\subsection{Layer 7: Collaboration and Processes}
\label{sec:iot-model-layer7}

\todo[inline]{To-do: Review section.}
% Não removia, uma secção curta e concisa está OK porque não é tão relevante para o trabalho como as outras.
% Podes mencionar, de forma conceptual apenas (não estou a ver nesta fase como operacionalizar no projecto % ainda, nem vala a pena especular) o seguinte:
% "Analyze opportunities to improve the efficiency and effectiveness of the processes."
% "Identification of the goals, deliverables, activities, patterns of collaboration, techniques, actions, and % technologies associated with each process."
% Ver:
% https://ieeexplore.ieee.org/document/6148656


The information that is created by the \acs{IoT} yields little value unless it prompts action, which requires people and processes (seventh layer). The objective is not the application — it is to empower people to work better and more efficiently. The sixth layer (Applications) provides business people the right insight, at the right time, so they can make the right decision. To do this people must be able to communicate and collaborate, which often requires multiple steps and transcends multiple applications \cite{Cisco2014}.

\section{Survey on \acs{IoT} Applications for Healthcare}
\label{sec:sim-approaches}


\todo[inline]{To-do: Complete section! 
- Add introductory paragraph! 
Perform a critical analysis of all related work.
You should make clear and focus on what is your contributions, and state clearly what is *not* your contributions (i.e. out of the scope of this work).
Falta um parágrafo introdutório.
O que se vai fazer nesta sub.secção? pq?
}
\todo[inline]{To-do: Complete section.}


% <---- Real-time tracking systems ---->

%==========================================================================================================================================
%[1] P. Fuhrer and D. Guinard, “Building a smart hospital using RFID technologies,” Eur. Conf. eHealth 2006, Proc. ECEH 2006, pp. 131–142, 2006.
%
%Contribution:
%- This paper presents one of the first IoT applications for healthcare services. The authors propose a real-time tracking system using RFID tags, which can track hospital equipment, staff, patients and medical files, which can minimize the risks of patient misidentification, loss / theft of assets and even drug counterfeiting. Thus, demonstrating how RFID technology can help improve healthcare services.
%
%Significance: 
%- The authors show how the proposed system can help improve healthcare services by mitigating the risks of patient misidentification, loss / theft of assets and even drug counterfeiting. 
%
%Future Work:
% "One must be sure that the deployment of radio frequency devices does not interfere with pacemakers, heart monitors or other electrical devices that are common in an hosptial. Furthermore, the consequences and side-effects of radio waves on the exposed humans have to be clarified. "
% "(...) it should be clear that challenging cryptographic issues are raised in relation with wireless transmission and that there is a need for clear laws and recommandations about the tracking of goods and people."
%
%- Analyse compatibility with other hospital equipments, and consequences of long term exposure to RF. 
%- Improving security on wireless transmissions;
%- Increase regulation regarding the tracking of goods and people;


Fuhrer et al. \cite{Fuhrer2006} describes one of the first IoT applications for healthcare. The authors propose a real-time locating system (RTLS) using \acs{RFID} tags. These tags are placed in hospital equipment, staff, patients and medical files and their location can be tracked through \acs{RFID} readers fixed in strategic locations around the hospital (\textit{e.g.} entrance of rooms, handheld readers). The authors demonstrate how systems like these can mitigate the risks of patient misidentification, loss or theft of assets and even drug counterfeiting. 


%==========================================================================================================================================
%[2] T. Adame, A. Bel, A. Carreras, J. Melià-Seguí, M. Oliver, and R. Pous, “CUIDATS: An RFID–WSN hybrid %monitoring system for smart health care environments,” Futur. Gener. Comput. Syst., vol. 78, pp. %602–615, Jan. 2018, doi: 10.1016/j.future.2016.12.023.
% 
%--------------------------
% Measured Signals: (L1) 
% - 
% Networking Protocols: (L2)
% - WiFi(868 MHz), RFID
% Edge Computing: (L3)
% - Fall Detection is done on IoT node, RTT on gateway
% Data Storage: (L4)
% - MySQL
% Data Formats: (L5)
% - Custom JSON format (No specification)
% Application Features: (L6)
% - Real-Time Tracking System (Patients and Assets), Fall Detection, Vital signs monitoring 
% Security:
% - AES-128 (iot node <-> gateway), WPA-Personal (gateway <-> server)
% Integration with HIS:
% - No
% Other Notes: 
% - Ran a hospital trial
%------------------------------
% Contribution:
%"- Design and implementation of a hybrid network consisting of RFID tags and readers, WSN beacons, and a gateway acting as a common element between both wireless technologies.
% - Development of an RTLS system running over the hybrid network to accurately locate patients and assets.
% - Development of an electronic wristband to track patients and monitor their vital signs within a health care environment."
% 
% Significance: 
%(...)
%
% Future Work:
%(...)
%-->

Adame et al. \cite{Adame2018} propose an IoT hybrid monitoring system - CUIDATS - for health care environments which integrates RFID and WSN technologies in a single platform providing real-time location, status, and tracking of patients and hospital assets. The patients are monitored via a small wristband which holds a small low power mobile sensor node, equipped with temperature, pulse and accelerometers.


%==========================================================================================================================================
%[3] T. Wu, F. Wu, C. Qiu, J. M. Redoute, and M. R. Yuce, “A Rigid-Flex Wearable Health Monitoring Sensor Patch for IoT-Connected Healthcare Applications,” IEEE Internet Things J., vol. 7, no. 8, pp. 6932–6945, 2020, doi: 10.1109/JIOT.2020.2977164.
%
%
%Notes:
%(...)
%
%----
%Contribution:
%"In this article, an innovative low-power wearable sensor patch is proposed for IoT-connected remote long-term healthcare applications. The sensor patch system consists of three main parts: 1) a center board for signal acquisition, processing, and transmission; 2) a power board for energy supply and charging batteries; and 3) different sensors for physiological parameters measurements. All components of the sensor patch is connected and presented in a rigid-flex structure, which is suitable for wearable health monitoring of ECG, PPG, HR, and body temperature. As the ECG and PPG are integrated on the same device, continuous BP estimation based on the PAT method can be achieved without extra wires or hardware configurations. The experimental results demonstrate the performance of the proposed sensor patch against the comparison with a commercial reference medical equipment."
%
%- Development of a innovative wearable sensor patch
%
%Significance: 
%(...)
%
%
%Future Work:
%"Since security is not the focus of this article, the two common security measures are implemented to meet the basic requirements of the following: (...) Security Between Wearable Patches and Gateways (...) Security Measures in Gateways and Cloud Server"
%"In our future work, more edge computing functions on the gateway will be developed for an IoT-connected healthcare platform."

Wu et al. \cite{Wu2020} developed a system which uses wearable sensor networks to monitoring the patients' status. The wearable sensors transmit the different physiological signals (ECG, PPG and body temperature) using BLE to gateways, which can either by fixed or mobile, by using smartphones. The gateway exchanges data with the cloud through bridged MQTT brokers, allowing the development of local features (e.g. local UI to interact with the patients) and cloud processing features (e.g. Big Data Analytics, data storage, UI for medical professionals). 

%==========================================================================================================================================
% [4] e-CoVig
%
% Contribution:
%
% Significance:
%
%
% Future work:
%
Recently, and motivated by the recent pandemic crisis, investigators from ISR-Lisboa developed a system called e-CoVig, a low-cost solution for monitoring patients during the COVID-19 quarantine. 

%==========================================================================================================================================
%[5] L. Catarinucci et al., “An IoT-Aware Architecture for Smart Healthcare Systems,” IEEE %Internet Things J., vol. 2, no. 6, pp. 515–526, Dec. 2015, doi: 10.1109/JIOT.2015.2417684.
%
%Notes:
%(...)
%
%----
%Contribution:
%(...)
%
%Significance: 
%(...)
%
%Future Work:
%(...)
%----


%..placeholder...

\subsection{Comparative Analysis}
% Baptiza cada trabalho com um nome teu, seguido da referencia.
 \renewcommand{\arraystretch}{2}
  \begin{sidewaystable}[h]
      \centering
      \begin{tabular}{l|l|l|l|l|l|l}
        \textbf{References} & \makecell{\textbf{Measured} \textbf{Signals}} & \makecell{\textbf{Networking} \textbf{Protocols}}& \makecell{\textbf{Data} \textbf{Storage}} & \makecell{\textbf{e-Health} \textbf{Standards}} & \textbf{Features} & \textbf{Security Features} \\ \hline
          \cite{Fuhrer2006} & \xmark & \makecell{EPC/RFID,\\ Wi-Fi} & MySQL & None & \makecell{RTLS}& \makecell{Unspecified Storage \\Encryption} \\
          \cite{Adame2018} & \makecell{Temperature, \\Heart Rate,\\ Accelerometer} & \makecell{EPC/RFID,\\ Wi-Fi} & MySQL & None & \makecell{RTLS, \\ Fall Detection,\\ Vital signs\\ monitoring}& \makecell{Unspecified Storage \\Encryption} \\
          % \cite{Catarinucci2015} & \makecell{Temperature, \\ \acs{ECG}, Accelerometer,\\  Barometric Pressure,\\  Ambient Light} & \makecell{EPC/RFID, 6LowPAN,\\ CoAP} & MySQL & None & RTLS & ? \\
          \cite{Wu2020} & \makecell{Temperature, \\Heart Rate,\\ Accelerometer} & \makecell{BLE, Wi-Fi, \\ MQTT} & MySQL & None & \makecell{RTLS, \\ Fall Detection,\\ Vital signs\\ monitoring}& \makecell{AES-128} \\
          \cite{Doukas2012} & \makecell{Temperature, \\Heart Rate,\\ } & & & & &  \\
          %\cite{} & & & & & &  \\
        \end{tabular}
      \caption{Comparison between different pervasive healthcare applications.}
      \label{tab:comparsion-articles}
  \end{sidewaystable} 
 \renewcommand{\arraystretch}{1}

\clearpage
\subsection{Weaknesses of literature}
\todo[inline]{To-do: Review section later!}
\label{sec:weaknesses}

% Apenas identificas o problema da Interoperabilidade e Standards?
% Pensa tb nas questões de segurança e escalabilidade. 
% Pode não ser uma fraqueza/desafio comum a *todos* os trabalhos na literatura mas ser algo q nem sempre é bem resolvido.

%\subsubsection{Security and Privacy} 
% From the literature, we find that researchers ensure data security and privacy through 
%\subsubsection{Interoperability} 

Despite recent efforts, interoperability is still an issue of \acs{IoT} systems. Due to the lack of clear and concise industry standards and regulations, many manufacturers develop their own proprietary data formats and communication protocols, which hampers the integration of new resources since the systems are designed within closed ecosystems \cite{Rubi2019}. Moreover, the adoption of new systems can be often met with much objection from the clinical staff due to their mistrust of technology \cite{DursunErgezen2020}. 
To facilitate the deployment of new healthcare systems in hospitals, these need to be integrated easily in existing \acs{HIS}. \bigskip

Fortunately, there are several international initiatives to promote the use of \acs{IoT} in health in a standardized way, such as HIMSS (Healthcare Information and Management Systems Society) and the Personal Connected Health Alliance (PCHAlliance). PCHAlliance for example promotes the adoption of the Continua Design Guidelines (CDG), which facilitates the integration of personal health devices into health systems. These guidelines have been recognized by ITU (International Telecommunication Union) and the European Commission and are adopted by countries such as Denmark, Norway and the USA, among others \cite{PersonalConnectedHealthAlliance2017}. These guidelines describe a series of e-health standards like \acs{FHIR} which facilitate the exchange of information between systems, in order to ensure the implementations become truly interoperable.

% Within this project we aim to continue to expand the capabilities of the Continua Design Principles tofurther extend the software to collect the data that is being exchanged between sensors, gateways, and end services to make implementations truly interoperable.

% The PCHAlliance publishes and promotes the global adoption of standards and the implementation guidelines that unleash the massive amounts of medical-grade data that enables a more holistic perspective. Commercial ready software enables the rapid integration of these standards into your product. A conformity assessment program verifies that the standards have been properly and uniformly implemented.

\section{Statement of Contributions}
\todo[inline]{To-do: Complete section!}
%tendo em conta o estado da arte, as suas limitações e pontos fortes, propomos um sistema que
%1) vai para alem do estado da arte nisto e naquilo.
%2) inspira-se nos conceitos extistentes nisto e naquilo.

% Note to self: The biostickers will be mentioned in the first chapter when presenting the objectives.
After studying the different approaches taken by researchers, and in context of the dissertation, we propose a novel fully modular \acs{IoT} infrastructure that uses the \acs{FHIR} standard in order to fully integrate the data in the existing \acs{HIS}, Glintt GlobalCare. Other researchers in the \acf{ISR} already developed wearable devices, designated ``biostickers'', using two different networking stacks — \acs{BLE} and \acs{EPC/RFID}. The system must ensure reliable and secure communications with the devices for each protocol, implementing the aforementioned security features in section \ref{sec:weaknesses}. From a hardware perspective, the system is composed of 3 different components, as seen Figure \ref{fig:wow-architecture}: the biostickers, that acquire the patient's physiological signals, the ``SmartBox'', which acts as a central node of the WBAN and aggregate the data and then communicates it to the gateway, and the ``Smart Gateway'', which serves as a fog server in order to mitigate latency and other computing issues and acts as the gateway to the \acs{HIS}. 

\begin{figure}[H]
    \centering
    \includegraphics[width=\linewidth]{images/wow-architecture.png}
    \caption{Overview of the proposed \acs{IoT}-connected healthcare system.}
    \label{fig:wow-architecture}
\end{figure}

For the work developed throughout the dissertation, we have set the following goals: 

\begin{itemize}
    \item Develop and deploy SmartBoxes embedded in hospital beds for data acquisition from biostickers attached to patients’ skin via \acs{RFID}:
    \begin{itemize}
        \item Hardware evaluation of 2 different \acs{IoT} kits (Raspberry Pi and Udoo Bolt).
        \item Viability study of a disruptive battery-less \acs{RFID} data acquisition compared to a \acs{BLE} acquisition.
        \item Implementation of reliable and secure \acs{RFID} data acquisition and general communication between the biosticker and the Smart Boxes.
        \item Selection of hardware components and assembly of SmartBox prototype;
    \end{itemize}
    \item Establishment of data integration pipelines using \acs{MQTT} and management of the multiple SmartBoxes in the Smart Gateway;
    \item Development of a \acs{FHIR} \acs{API} layer to integrate the proposed system in the GlobalCare \acs{HIS}.
    \item Evaluation of the performance of the proposed system through controlled lab tests and later deployment hospital trials within the WoW project.
\end{itemize}
\todo[inline]{Adiciono também o objetivo da análise de dados? Ou até ver deixo só estes?}

\section{Summary}

In the next chapter, we begin with the hardware evaluation of the different \acs{IoT} kits and the implementation of secure communications between the biostickers and the SmartBoxes.

% - Hardware evaluation for edge nodes which integrate electronic wireless patches that gather patient's physiological signals;
% - Integrating IoT system in an existing healthcare information system (Glintt GlobalCare software) through an FHIR API layer;
% - Evaluation of the performance of the proposed system through a testbed and a real healthcare scenario;

% In the context of the WoW project, a preliminary study will be performed, by analyzing 24 hours of data from 3 volunteers who are using a specific drug (i.e.! paracetamol) in order to illustrate the effect of the drug intake on various parameters, including temperature, emotions, heart rate, blood oxygen, and respiration.

% Se fizer sentido, na dissertação podes ter um capitulo de "Technological Background", que faz um overview às ferramentas de sotware e ao próprio hardware utilizado no projecto. (to be discussed later)

\bigskip