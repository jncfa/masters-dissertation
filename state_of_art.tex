In this chapter a survey of pervasive healthcare applications is presented. In order to gain a greater understanding of which are the building blocks of an \acf{IoT} system, a reference model is also presented.

\todo[inline]{To-do: Add small chapter introduction. }

\section{Internet of Things}

\subsection{What is IoT?}

\acl{IoT} (or \acs{IoT}) is an emerging communication paradigm, often hailed as the driver of the Fourth Industrial Revolution \cite{Aceto2020}. \bigskip

The definition of this concept has evolved over time with the development of other technologies such as data analytics, embedded systems, etc. Nowadays it describes a strategy supported on the development of networks of smart devices that exchange and process information through Machine-to-Machine (M2M) communications, usually based on the Internet Protocol (IP). This technology enables ubiquitous systems to gather remarkable amounts of information regarding the surrounding environment, which can later be turned into insight through the usage of data analytics tools, like Machine Learning algorithms \cite{Singh2019}. \bigskip

More specifically in the healthcare domain, 

\todo[inline]{To-do: Discuss the potential of IoT technologies and bridge to pervasive healthcare applications (such as in Clinics, Hospitals, Smart Home). Cite articles with references to investments in these areas.

- smart systems enable continous patient monitoring
}

\section{An IoT Reference Model for Pervasive Healthcare Applications}

In order to develop an IoT system, it is crucial to design it based on a reference model. A reference model provides a general structure (or a ``template'') for designing systems, thus enabling the comprehension of these complex systems by breaking them down into simple and distinct functional layers, as also defines some common terminology used in its domain.\bigskip

In 2014 the \acs{IoT} World Forum (IoTWF) architectural committee published an \acs{IoT} architectural reference model, composed by seven layers as shown in figure \ref{fig:iotwf-referencemodel}. This model provides a simple and clean functional view into the different components of an \acs{IoT} system without restricting the scope or locality of its components, enabling both edge / fog computing solutions and more cloud oriented approaches. However, from a hardware perspective, in this work we will restrict our focus to the most common approach taken by investigators, using 3 different components: 

\begin{itemize}
    \item \textbf{Endpoint} or \textbf{edge} nodes (corresponding to Layer 1), which interact with the physical world, capturing data.
    \item \textbf{Gateway} devices (Layers 2-3), which connect to multiple \textbf{edge} nodes, filtering and aggregating the data generated by these, while relaying it to a central server; 
    \item \textbf{Central} server (Layers 4-6), which is responsible for collecting, storing and analyzing the captured data in order to provide users with valuable insight;
\end{itemize}

While this model can be used to develop IoT systems for any industry (from agriculture to smart cities), in the context of the dissertation we will focus on pervasive healthcare applications and its enabling technologies. 

% \todo[inline]{To-do: Make images that summarize key points of each layer (by adapting original images from Cisco documentation?)}

\begin{figure}[H]
    \centering
    \includegraphics[width=0.85\linewidth]{images/iotwf-referencemodel.png}
    \caption[IoT reference model published by IoTWF.]{IoT reference model published by IoTWF. Source: \cite{Cisco2014}.}
    \label{fig:iotwf-referencemodel}
\end{figure}

\subsection{Layer 1: Physical Devices and Controllers}
\label{sec:iot-model-layer1}

The first layer of the model is the physical devices and controller layer. This layer houses the ``things'' in the \acl{IoT}: the endpoint devices composed of sensors and actuators that perceive and interact with the physical world. Through those interactions, the devices generate data, which is then sent across the network for analysis and storage. \bigskip

When designing an \acs {IoT} network, the first step should be to analyze the mobility and data requirements of the system \cite{10.5555/3161403}. \textbf{Mobility} describes the devices' ability to move and, if it is able to, how frequently it does so. \textbf{Data} requirements describe how much data is generated and transmitted by each device per unit of time, and how critical is it to the operation. Simpler health monitoring applications can include a single temperature or heart rate sensor while more complex applications can include pulse oximetry, electrocardiogram (\acs{ECG}), respiration rate, etc. \bigskip

With these key requirements established, we can now discuss some other characteristics of the smart devices, like:

\begin{itemize}
    \item \textbf{Power source}: This classification describes if the device has an internal energy supply powering the device or if it has continuous power delivery from an external source. If the device must be mobile, it will require a portable power source, a battery. Battery-powered devices are not bound to a single location, but the finite energy source constrains the device's energy consumption and lifetime, leading to limited memory, computation and connectivity capabilities. 
    \item \textbf{Transmission range}: This classification describes how far away the devices can communicate. In healthcare, these usually have very short transmission ranges. For example, a fitness band that communicates with a smartphone will be at most located a few meters from it.
\end{itemize}

In order to properly design these devices, it is necessary to understand what problems currently reside within clinical environments. Today, hospitalized patients need to be wired to various measurement instruments when continuous biomonitoring is required. This confines the patients to their beds, restricting their mobility, and may also cause skin irritations and infections, aggravating their discomfort and deterioration of their health condition \cite{Darwish2011}. Moreover, the detachment of electrodes from the patient's body, provoked by patient's movements, is one of the main sources of false alarms. These require immediate attention from the clinic's staff, contributing to their exhaustion and may ultimately result in the desensitization to the alarms, reducing their response time to real emergencies \cite{DursunErgezen2020}. Wearable, wireless, and non-intrusive devices can to minimize these issues to a large extent, and are seen as one of the key components of \acs{IoT}-based healthcare systems \cite{Baker2017}.  \bigskip

In recent years there has been remarkable progress on the development of wearable devices, driven by recent technological breakthroughs in the miniaturization of sensors and microfabrication processes \cite{Adame2018}. From the literature, we can classify the sensors used in these devices in 3 distinct categories based on information that can be extracted from them, as shown in table \ref{tab:layer1-sensors}:

\begin{itemize}
    \item Monitoring the patient's biosignals, used for evaluating the patient's health condition.
    \item Monitoring the patient's activity or motion, used for detecting fall events, determining the patient's location and travelled distance, estimating the patient's body posture, etc.
    \item Monitoring the patient's environment, mainly used for assessing environmental hazards, \textit{e.g.} gas leaks in a patient's home or an industrial workplace.
\end{itemize}

\begin{table}[H]
    \centering
    \begin{tabular}{l|l}
        \textbf{Type of monitoring} & \textbf{Type of sensors used} \\ 
        \hline
        \multirow{3}{*}{Vital Signs Monitoring} & Body Temperature, Heartbeat, Respiratory Rate, \\ & Galvanic Skin Response, Blood Pressure, Pulse Oximetry, \\ & \acf{ECG}, Glucose Level Sensors \\ \hline
        \multirow{2}{*}{Activity Monitoring} & Accelerometer, Gyroscope, Magnetometer, Ultrasound Sensors,\\ & \acf{RFID} Tags\\
        \hline
        \multirow{1}{*}{Environmental Monitoring} & Air Temperature, Humidity, Hazardous Gas Sensors \\
    \end{tabular}
    \caption[List of sensors commonly used in pervasive healthcare applications]{List of sensors commonly used in pervasive healthcare applications, adapted from \cite{MinhDang2019}.}
    \label{tab:layer1-sensors}
\end{table}

% Current research focuses on minimizing energy consumption of the devices to extend their lifetime. (...)

\subsection{Layer 2: Connectivity}
\label{sec:iot-model-layer2}

\todo[inline]{To-do: Remake this section!!!!}

The second layer of the model focuses on connectivity, on linking the different components of our system. The most important function of this layer is provide reliable and timely data transmission. This includes all communications within the system, which can be split into two categories: communications within the local network (\textit{e.g.} between edge nodes and the gateway devices), and communications between the edge of the local network (\textit{e.g.} gateway devices) and the central server. \bigskip

\subsubsection{Communication Protocols}

Technologies are designed with certain use cases in mind. They drive their development and thus it is natural that each one has their own advantages and disadvantages, depending on their use. For instance, short range wireless protocols like Wi-Fi or Bluetooth are limited by the transmission range, but long range protocols like NB-IoT usually have high energy consumption, which are unviable for networks with constrained devices. Each protocol also defines their own frame format and communicate within certain frequency bands, which may require special licenses. Due to the frequency bands used by UHF \acs{RFID}, metal and water surfaces near the devices introduce great interference in its transmissions due to signal reflections, thus reducing the effective transmission range \cite{Cairo2018}. The figure \ref{fig:communication-protocols} shows some commonly used protocols in \acs{IoT} systems grouped by range. 

\begin{figure}[H]
    \centering
    \includegraphics[width=0.85\linewidth]{images/communication-protocols.png}
    \caption[Classification and grouping of various network protocols by range.]{ Classification and grouping of various network protocols by range. Source: \cite{10.5555/3161403}}
    \label{fig:communication-protocols}
\end{figure}

\todo[inline]{To-do: Make new image based on this one.}

The choice of the communication protocol is driven by the characteristics of the smart devices, as defined in the first layer. However, we can highlight other key points that affect this decision:

\begin{itemize}
    %\item \textbf{Cost}: The cost of implementing certain protocols may not be economically viable. \acs{RFID} systems are a great example of this. The development of \acs{RFID} tags is inexpensive, but this is offset by the immense cost of the \acs{RFID} readers, which can quickly become unsustainable.
    \item \textbf{Latency}: Certain applications deal with time critical events, for example the detection of health emergencies. In these cases, any delays in the communications can cause great detriment to the patient's well-being, making it crucial to minimize them.
    \item \textbf{Throughput}: The communication protocol should ensure there is enough bandwidth to handle all communications within the designated transmission range. Even within similar technologies, this can vary wildly with the range as seen in figure \ref{fig:communication-protocols-throughput}. 
    \item \textbf{Security}: Security is one of the most important requirements of any system, but this is especially true for healthcare systems. Due to the sensitive nature of the information, it is crucial to secure the information from malicious actors. Communication protocols must implement security mechanisms, such as encryption or data integrity verifications, that ensure the transmissions are not compromised in transit, thus denying third parties the ability to snoop or tamper the transmissions.
    \item \textbf{Interoperability}: To ensure the interoperability of the system it is imperative to choose protocols that are widely accepted and supported by the industry. %, ensuring maintainability.  
    \item \textbf{Scalability}: This determines how many devices are supported and how many more can be added to the system, thus giving us a measure of the system's flexibility for expanding beyond the initial development. 
\end{itemize}
 
\begin{figure}[H]
    \centering
    \includegraphics[width=0.55\linewidth]{images/communication-protocols-throughput.png}
    \caption{Throughput versus Transmission range for four WHAN to WLAN communications protocols. Source: \cite{10.5555/3161403}}
    \label{fig:communication-protocols-throughput}
\end{figure}

\todo[inline]{To-do: Complete section with a small comparison of most common protocols used in healthcare? BLE, ZigBee?}

% \subsection{Security in the IoT}
\todo[inline]{To-do: Add a section exposing most common security measures and attack vectors?}

% \acs{IoT}
% Discussions of security for each level and for the movement of data between levels could fill a multitude of papers.
% For the purpose of the IoT Reference Model, security measures must:
% ● Secure each device or system
% ● Provide security for all processes at each level
% ● Secure movement and communication between each level, whether north- or south-bound

\subsection{Layer 3: Edge (Fog) Computing}
\label{sec:iot-model-layer3}

\acs{IoT} systems can often have hundreds or even thousands of sensors generating data multiple times per second, 24 hours per day, which can demand an unsustainable amount of network and computing resources. Moreover, certain applications may be time critical, where delays in communication can be very detrimental. Therefore, it is crucial to initiate data processing as close to the edge of the network as possible. This paradigm is usually referred to as ``edge computing'' (when the data processing occurs at the endpoint devices) or ``fog computing'' (when it happens at the edge of local network, \textit{e.g} in ``gateway'' devices). The third layer of the model defines how the system prepares the generated data for storage and higher level processing for the next layers. The data processing at this stage is generally very limited, mostly focused on ``preprocessing'' the data and handling time critical events. More demanding and thorough analysis should be left to the central server, which holds much greater computing power. The different processes applied at this stage usually are:

\begin{itemize}
    \item \textbf{Filtering}: Assessing if the data should be processed at a higher level. 
    \item \textbf{Formatting}: Reformatting data to ensure consistent formats for higher-level processing.
    \item \textbf{Cleaning}: Reducing data to minimize the impact of data on the network and higher level processing systems.
    \item \textbf{Analysis}: Determining whether data represents a threshold or alert. This is especially relevant for applications that deal with time critical events as seen in the previous section.
\end{itemize}

\subsection{Layer 4: Data Accumulation}
\label{sec:iot-model-layer4}

\todo[inline]{To-do: Review this section later.}

The data that is generated by the edge devices is propagated through the network, moving through each layer with each sensor reading. Up to this point, the model is event driven. However, most applications cannot make use of the data as it is generated. Typically, it is assumed the data is at rest and can be easily accessed. In this layer, Data Accumulation, we define how the system captures the data and stores it, so it becomes usable for applications when needed, providing a transition to query-based processing. This includes any issues related with data storage: how the data should be stored (\textit{e.g.} using a relational database, non-relational database, distributed file systems, etc.), what data should be stored and which should be kept for short-term use, etc.

In healthcare, big data analytics is seen as one of the most promising features of \acs{IoT} applications. However, the massive data collection that is associated with ubiquitous systems causes many issues concerning the processing and storage of data. Cloud platforms are often seen as a solution to this problem \cite{Baker2017}. This is made possible due to the elasticity in allocating, swiftly and inexpensively, shared resources on-demand, adjusting itself to the needs of each application. We can find 3 distinct types of cloud services:

\begin{itemize}
    \item \textbf{Infrastructure as Service (IaaS)}: Provides control over the remote machine (composed of virtual or dedicated hardware), operative system and middleware. This approach gives system designers the highest level of flexibility over the infrastructure.
    \item \textbf{Platform as a Service (PaaS)}: Provides a simple framework for developing applications, where the service provider manages the underlying infrastructure issues such as software updates and hardware maintenance. 
    \item \textbf{Software as a Service (SaaS)}: Provides the finished applications used by the end users, in this case health workers, that enable them to work. A simple example is a web-based email service, such as Gmail or Microsoft Outlook. 
\end{itemize}

\begin{figure}[H]
    \centering
    \includegraphics[width=\linewidth]{images/cloud-services.png}
    \caption[Differences between the cloud offerings and on-premise solutions.]{ Differences between the cloud offerings and on-premise solutions. Source: \cite{RedHat2021}}
    %\label{fig:communication-protocols}
\end{figure}

\subsection{Layer 5: Data Abstraction}
\label{sec:iot-model-layer5}

\todo[inline]{To-do: Complete section.}

% Reconciles multiple data formats and ensures consistent semantics from various sources. Confirms that the data set is complete and consolidates data into one place or multiple data stores using virtualization.
% The fifth layer is tasked with reconciling the 

% IoT systems will need to scale to a corporate—or even global—level and will require multiple storage systems to accommodate IoT device data and data from traditional enterprise ERP, HRMS, CRM, and other systems. The data abstraction functions of Level 5 are focused on rendering data and its storage in ways that enable developing simpler, performance-enhanced applications. With multiple devices generating data, there are many reasons why this data may not land in the same data storage:
% ● There might be too much data to put in one place.
% ● Moving data into a database might consume too much processing power, so that retrieving it must be separated from the data generation process. This is done today with online transaction processing (OLTP) databases and data warehouses.
% ● Devices might be geographically separated, and processing is optimized locally.
% ● Levels 3 and 4 might separate “continuous streams of raw data” from “data that represents an event.” Data storage for streaming data may be a big data system, such as Hadoop. Storage for event data may be a relational database management system (RDBMS) with faster query times.
% ● Different kinds of data processing might be required. For example, in-store processing will focus on different things than across-all-stores summary processing.
% For these reasons, the data abstraction level must process many different things. These include:
% ● Reconciling multiple data formats from different sources
% ● Assuring consistent semantics of data across sources
% ● Confirming that data is complete to the higher-level application
% ● Consolidating data into one place (with ETL, ELT, or data replication) or providing access to multiple data stores through data virtualization
% ● Protecting data with appropriate authentication and authorization;
% ● Normalizing or denormalizing and indexing data to provide fast application access;

\subsubsection{\acf{FHIR}}

As patients continuously move around the healthcare ecosystem, their health information must be available, discoverable and understandable to different entities (hospitals, laboratories, pharmacies, etc.). This prompts the digitization of medical files and the development of standards for exchanging these records instantly and securely to authorized users \cite{HL72019}, which are called \acl{EHR}s (\acs{EHR}s). \acs{EHR}s is the digital equivalent of a patient's paper-chart, it contains the patients' full medical history: previous diagnoses, treatment plans, test results, known allergies, among other details. It is now an essential component of health IT. 

\acf{FHIR} is a standard data format for exchanging \acs{EHR}s, developed by Health Level Seven International (HL7). HL7 is a non-profit organization involved in the development of international healthcare informatics for over 20 years. \acs{FHIR} builds upon previous data format standards like HL7 v2 and HL7 v3, and is widely adopted within the healthcare industry. 

\subsection{Layer 6: Application}
\label{sec:iot-model-layer6}

\todo[inline]{To-do: Complete section.}

% Interprets data using software application. Applications may monitor, control and provide reports based on the analysis of the data.

The sixth layer is the application layer, where the interpretation of the captured data occurs. 

% Finally, machine learning to perform diagnostics or provide treatment plans would be extremely valuable in a healthcare context, so a cloud storage framework for healthcare would need to enable value. As all of the characteristics of big data are important to healthcare applications, recent research in this area has focused on storing a wide variety of data generated by voluminous IoT systems in an organized manner that may be useful for later data analysis.

% Centralized and hassle-free collection of data from humans, is a very desired topic in digital health, since it would allow discovery of new digital biomarkers. That is, by acquisition and analysis of electrical/auditory/other physical events of the body, one can find new relationships between certain conditions of the patients and these events [76]. As an example, it is known that dementia has effects on regulation of the body temperature. Therefore, would it be possible to use continuous monitoring of the body temperature and AI to discover Alzheimer development? Although this is not the objective of this project, the proposal makes an important step towards providing large and diverse data to data scientists for analysis. Within this project, we will demonstrate a preliminary example of this ambitious objective, by analyzing 24 hours of data from 3 volunteer who are using a specific drug (i.e. paracetamol or similar), and demonstrate the effect of the drug intake on various parameters, including temperature, emotions, heart rate, blood oxygen, and respiration.

% Level 6 is the application level, where information interpretation occurs. Software at this level interacts with Level 5 and data at rest, so it does not have to operate at network speeds.
% The IoT Reference Model does not strictly define an application. Applications vary based on vertical markets, the nature of device data, and business needs. For example, some applications will focus on monitoring device data. Some will focus on controlling devices. Some will combine device and non-device data. Monitoring and control applications represent many different application models, programming patterns, and software stacks, leading to discussions of operating systems, mobility, application servers, hypervisors, multi-threading, multi-tenancy, etc. These topics are beyond the scope of the IoT Reference Model discussion. Suffice it to say that application complexity will vary widely.
% Examples include:
% ● Mission-critical business applications, such as generalized ERP or specialized industry solutions
% ● Mobile applications that handle simple interactions
% ● Business intelligence reports, where the application is the BI server
% ● Analytic applications that interpret data for business decisions
% ● System management/control center applications that control the IoT system itself and don’t act on the data produced by it.
% If Levels 1-5 are architected properly, the amount of work required by Level 6 will be reduced. If Level 6 is designed properly, users will be able to do their jobs better. Figure 8 depicts Level 6.

\subsection{Layer 7: Collaboration and Processes}
\label{sec:iot-model-layer7}

\todo[inline]{To-do: The Cisco model defines a seventh layer: Collaboration and Processes, but how can one model for it (if it is even possible)? Should this section be removed entirely, or what should be added to make it more complete? }

The information that is created by the \acs{IoT} yields little value unless it prompts action, which requires people and processes (seventh layer) — this is what differs \acs{IoT} from traditional \acl{IT} systems. The objective is not the application — it is to empower people to work better and more efficiently. The sixth layer (Applications) provides business people the right insight, at the right time, so they can make the right decision. To do this people must be able to communicate and collaborate, which often requires multiple steps and transcends multiple applications \cite{Cisco2014}.

\section{Similar approaches}

\todo[inline]{To-do: Complete section.}


% <---- Real-time tracking systems ---->

%==========================================================================================================================================
%[1] P. Fuhrer and D. Guinard, “Building a smart hospital using RFID technologies,” Eur. Conf. eHealth 2006, Proc. ECEH 2006, pp. 131–142, 2006.
%
%Contribution:
%- This paper presents one of the first IoT applications for healthcare services. The authors propose a real-time tracking system using RFID tags, which can track hospital equipment, staff, patients and medical files, which can minimize the risks of patient misidentification, loss / theft of assets and even drug counterfeiting. Thus, demonstrating how RFID technology can help improve healthcare services.
%
%Significance: 
%- The authors show how the proposed system can help improve healthcare services by mitigating the risks of patient misidentification, loss / theft of assets and even drug counterfeiting. 
%
%Future Work:
% "One must be sure that the deployment of radio frequency devices does not interfere with pacemakers, heart monitors or other electrical devices that are common in an hosptial. Furthermore, the consequences and side-effects of radio waves on the exposed humans have to be clarified. "
% "(...) it should be clear that challenging cryptographic issues are raised in relation with wireless transmission and that there is a need for clear laws and recommandations about the tracking of goods and people."
%
%- Analyse compatibility with other hospital equipments, and consequences of long term exposure to RF. 
%- Improving security on wireless transmissions;
%- Increase regulation regarding the tracking of goods and people;


Fuhrer et al. \cite{Fuhrer2006} describes one of the first IoT applications for healthcare. The authors propose a real-time locating system (RTLS) using \acs{RFID} tags. These tags are placed in hospital equipment, staff, patients and medical files and their location can be tracked through \acs{RFID} readers fixed in strategic locations around the hospital (\textit{e.g.} entrance of rooms, handheld readers). The authors demonstrate how systems like these can mitigate the risks of patient misidentification, loss or theft of assets and even drug counterfeiting. 


%==========================================================================================================================================
%[2] T. Adame, A. Bel, A. Carreras, J. Melià-Seguí, M. Oliver, and R. Pous, “CUIDATS: An RFID–WSN hybrid %monitoring system for smart health care environments,” Futur. Gener. Comput. Syst., vol. 78, pp. %602–615, Jan. 2018, doi: 10.1016/j.future.2016.12.023.
% 
%--------------------------
% Measured Signals: (L1) 
% - 
% Networking Protocols: (L2)
% - WiFi(868 MHz), RFID
% Edge Computing: (L3)
% - Fall Detection is done on IoT node, RTT on gateway
% Data Storage: (L4)
% - MySQL
% Data Formats: (L5)
% - Custom JSON format (No specification)
% Application Features: (L6)
% - Real-Time Tracking System (Patients and Assets), Fall Detection, Vital signs monitoring 
% Security:
% - AES-128 (iot node <-> gateway), WPA-Personal (gateway <-> server)
% Integration with HIS:
% - No
% Other Notes: 
% - Ran a hospital trial
%------------------------------
% Contribution:
%"- Design and implementation of a hybrid network consisting of RFID tags and readers, WSN beacons, and a gateway acting as a common element between both wireless technologies.
% - Development of an RTLS system running over the hybrid network to accurately locate patients and assets.
% - Development of an electronic wristband to track patients and monitor their vital signs within a health care environment."
% 
% Significance: 
%(...)
%
% Future Work:
%(...)
%-->

Adame et al. \cite{Adame2018} propose an IoT hybrid monitoring system - CUIDATS - for health care environments which integrates RFID and WSN technologies in a single platform providing real-time location, status, and tracking of patients and hospital assets. The patients are monitored via a small wristband which holds a small low power mobile sensor node, equipped with temperature, pulse and accelerometers.


%==========================================================================================================================================
%[3] T. Wu, F. Wu, C. Qiu, J. M. Redoute, and M. R. Yuce, “A Rigid-Flex Wearable Health Monitoring Sensor Patch for IoT-Connected Healthcare Applications,” IEEE Internet Things J., vol. 7, no. 8, pp. 6932–6945, 2020, doi: 10.1109/JIOT.2020.2977164.
%
%
%Notes:
%(...)
%
%----
%Contribution:
%"In this article, an innovative low-power wearable sensor patch is proposed for IoT-connected remote long-term healthcare applications. The sensor patch system consists of three main parts: 1) a center board for signal acquisition, processing, and transmission; 2) a power board for energy supply and charging batteries; and 3) different sensors for physiological parameters measurements. All components of the sensor patch is connected and presented in a rigid-flex structure, which is suitable for wearable health monitoring of ECG, PPG, HR, and body temperature. As the ECG and PPG are integrated on the same device, continuous BP estimation based on the PAT method can be achieved without extra wires or hardware configurations. The experimental results demonstrate the performance of the proposed sensor patch against the comparison with a commercial reference medical equipment."
%
%- Development of a innovative wearable sensor patch
%
%Significance: 
%(...)
%
%
%Future Work:
%"Since security is not the focus of this article, the two common security measures are implemented to meet the basic requirements of the following: (...) Security Between Wearable Patches and Gateways (...) Security Measures in Gateways and Cloud Server"
%"In our future work, more edge computing functions on the gateway will be developed for an IoT-connected healthcare platform."

Wu et al. \cite{Wu2020} developed a system which uses wearable sensor networks to monitoring the patients' status. The wearable sensors transmit the different physiological signals (ECG, PPG and body temperature) using BLE to gateways, which can either by fixed or mobile, by using smartphones. The gateway exchanges data with the cloud through bridged MQTT brokers, allowing the development of local features (e.g. local UI to interact with the patients) and cloud processing features (e.g. Big Data Analytics, data storage, UI for medical professionals). 

%==========================================================================================================================================
% [4] e-CoVig
%
% Contribution:
%
% Significance:
%
%
% Future work:
%
Recently, and motivated by the recent pandemic crisis, investigators from ISR-Lisboa developed a system called e-CoVig, a low-cost solution for monitoring patients during the COVID-19 quarantine. 

%==========================================================================================================================================
%[5] L. Catarinucci et al., “An IoT-Aware Architecture for Smart Healthcare Systems,” IEEE %Internet Things J., vol. 2, no. 6, pp. 515–526, Dec. 2015, doi: 10.1109/JIOT.2015.2417684.
%
%Notes:
%(...)
%
%----
%Contribution:
%(...)
%
%Significance: 
%(...)
%
%Future Work:
%(...)
%----


%..placeholder...

\subsection{Comparative Analysis}
% Quadro comparativo com maiores diferenças entre cada projeto e o WoW-WP4

\todo[inline]{To-do: Place table with a list of criteria to compare the different approaches.}
% Competing solutions and constraints
%
% Most of our competitors are only partial — they either develop IoT platforms to collect the signals from other devices already in the market, or they develop innovative medical devices for remote monitoring. The majority of the companies commercializing telemonitoring solutions use big equipment/ medical devices manufactured by other companies, which use communication standards and Wi-Fi to transmit the data to a cloud platform developed by them, where it is analyzed, providing healthcare professionals the results. Mainly, these solutions are used for domiciliary care, not in hospital facilities, and some are specific for certain pathologies (diabetes, COPD). Examples of companies providing this type of solutions are: HopeCare and VitalMobile in Portugal and Tunstall in Spain. Other companies develop wireless wearables to collect biosignals, mainly for wellness purposes. Most of these devices cannot be used for diagnosing without further exams. An example is QardioCore, a wearable electrocardiogram monitor, designed to improve detection and monitoring of cardiac conditions during the daily basis and when performing physical activity. The data collected by the device is sent to an App, which analyses the signal. The results can then be sent by e-mail to the physician. This solution does not have a platform or an online portal to support the analysis, neither it integrates with other systems

\subsection{Weaknesses of literature}

\subsubsection{Security and Privacy}

\todo[inline]{To-do: Complete section.}

\subsubsection{Interoperability}

Despite recent efforts, interoperability is still an issue of IoT systems. Due to the lack of clear and concise industry standards and regulations, many manufacturers push their own proprietary data formats and communication protocols, which hampers the integration of new resources since they are developed within closed ecosystems \cite{Rubi2019}. 
Moreover, the adoption of new systems can be often met with much objection from the clinical staff due to their mistrust of technology \cite{DursunErgezen2020}. 
In order to facilitate the integration of new systems, these should be easily 



%In context of healthcare solutions, PCHAlliance...
% PCHAlliance https://www.pchalliance.org/continua-design-guidelines 
%\begin{figure}[H]
%    \centering
%    \includegraphics[width=.90\linewidth]{images/cdg-architecutre.png}
%    \caption{System architecture from the Personal Connected Health Alliance \cite{ContinuaHealthAlliance}.}
%    \label{fig:continua-architecture}
%\end{figure}

\section{Statement of Contributions}
%tendo em conta o estado da arte, as suas limitações e pontos fortes, propomos um sistema que
%1) vai para alem do estado da arte nisto e naquilo.
%2) inspira-se nos conceitos extistentes nisto e naquilo.

\todo[inline]{To-do: Complete section.}

After studying the different approaches taken by investigators, 
\begin{itemize}
    \item Hardware evaluation for edge nodes which integrate electronic wireless patches that gather patient's physiological signals;
    \item Integrating IoT system in an existing healthcare information system (Glintt GlobalCare software) through an FHIR API layer;
    \item Evaluation of the performance of the proposed system through a testbed and a real healthcare scenario;
\end{itemize}
