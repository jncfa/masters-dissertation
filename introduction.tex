%INTRODUCTION: deve responder a "WHAT?" e a "WHY?"

% This report describes the work developed under the project "Wireless IoT Architecture for Smart Nodes deployed in Hospital Beds", which took place in the Institute of Systems and Robotics (ISR) in Coimbra in the past year. 

\section{Context}

\paragraph{} Due to all the technological and healthcare advances over the last years, we have observed a steady increase of life expectancy. With the growing aging population a rise of chronic illnesses can be noticed, which places significant strain on modern healthcare systems due to their limited resources ~\cite{Koen2019, Redondi2013} - both human and material like medical equipment, hospital beds, etc. This is an evermore pressing concern, particularly with the recent Covid-19 epidemic, that is testing the limits of current healthcare systems. 

\paragraph{} In an effort to counter this, many countries and organizations are currently promoting the shift towards digital healthcare through several funding programs, such as European Union \cite{EuropeanUnion2021} and World Health Organization initiatives \cite{WorldHealthOrganization2020}. The usage of digital technologies in health has the potential to radically change how healthcare is delivered, by enhancing the efficiency and cost-effectiveness of care, and enabling new business models for service providers \cite{WorldHealthOrganization2020}.

\paragraph{} In particular, there is one paradigm which has stood out, having potential to fulfill this vision for digital health -- \textbf{\acf{IoT}}. \acs{IoT} has been used to revolutionize different industries, such as smart grids \cite{Faria2020}, oil and gas industry \cite{Shoja2018} and many more. The basic concept of \acs{IoT} is enabling processing capability and connectivity to ``physical objects'' or groups of these, allowing them to be capable of connecting with other devices and exchange data through the Internet \cite{gershenfeld2004internet}. Today, we find that hospitalized patients need to be wired to various measurement instruments when continuous biomonitoring is required. \acs{IoT} is not only capable of challenging this restriction, detaching patients from their beds and restoring their much-needed comfort (perhaps even allowing them to return to their own homes), but also provide value to the various stakeholders in the healthcare system with the automatization of processes, continuous and remote monitoring, clinical decision support, etc.

\paragraph{} However, there are still many challenges to tackle before deploying such technologies in a medical environment. In particular, interoperability, security, and privacy are challenges which are recurrently identified in the literature. 

Any device that is exposed to the Internet is a possible security liability, and thus the development of efficient security measures is crucial to ensure that data remains private. 

Moreover, the adoption of these novel systems can be often met with much objection from the clinical staff due to their mistrust of technology \cite{DursunErgezen2020}. To facilitate their deployment in hospitals, these need to be integrated easily in existing \acl{HIS}s (\acs{HIS}).

% \paragraph{} Moreover, today, we find that hospitalized patients need to be wired to various measurement instruments when continuous biomonitoring is required. This confines the patients to their beds, restricting their mobility, which can cause skin irritations and infections, aggravating their discomfort and deterioration of their health condition \cite{Darwish2011}. Furthermore, the detachment of electrodes from the patient's body, provoked by patient's movements, is one of the main sources of false alarms. These require immediate attention from the hospital staff, contributing to their exhaustion and may ultimately result in the desensitization to the alarms, reducing their response time to real emergencies \cite{DursunErgezen2020}.


% The \acf{WoW} project is a ambicious project by a consortium of different technological partners, such as Glintt, and academia, including the Coimbra Hospital and Universitary Centre (CHUC), Carnegie Mellon University (CMU-Portugal) and the Institute of Systems and Robotics (ISR). The overall goal is to demonstrate wireless patient bio monitoring and centralized data collection, processing and transmission, as a step towards domiciliary hospitalization. More specifically, the project foresees implementations of a novel material and system architecture that would allow for the first time, simultaneous monitoring of multiple patients through truly wireless and untethered thin-film biomonitoring stickers, via centralized data collection, processing and transmission to the most utilized software in the Portuguese hospitals, i.e. GlobalCare, a proprietary \acf{HIS} developed by the project leader – Glintt. 


% Context and Motivation.
% Expand on: Why digital healthcare is important/ IoT/ Areas of Application of IoT/Challenges
% Split different ideas into separate paragraphs.

\section{System Requirements}

Previous work by researchers at the Institute of Systems and Robotics (ISR) resulted in the development of innovative wearable devices, designated \textit{Biostickers}, which are electronic patches equipped with sensors that gather the patient's physiological signals and communicate wirelessly \cite{Silva2020}. 


\paragraph{} \textbf{The main objective of this dissertation work} is the development and validation of an \acs{IoT} architecture capable of integrating the data that is acquired by the \textit{Biostickers} into an existing \acs{HIS}, in the context of the \acs{WoW} R\&D project\footnote{WoW -- A step towards domiciliary hospitalization: \url{https://inovglintt.com/financiamento/wow/}}. This system serves as the ``backbone'' of the entire \acf{IT} system for the project, connecting these \textit{Biostickers} to the \acs{HIS} while addressing crucial issues like those described previously.

%\todo[inline]{To-do: Discuss importance of my work, both general impact and to the WoW project.}

\paragraph{} The requirements for the system, and their priorities based on a MoSCoW prioritization analysis \cite{stapleton1997dsdm}, are the following: 
% The system should be non-invasive, reliable and satisfy the stringent security and privacy requirements of health information systems. 

% - Hardware evaluation for edge nodes which integrate electronic wireless patches that gather patient's physiological signals;
% - Integrating IoT system in an existing healthcare information system (Glintt GlobalCare software) through an FHIR API layer;
% - Evaluation of the performance of the proposed system through a testbed and a real healthcare scenario;

% Main Aim and contributions: 
%- Summarise the problem
%- How in **general terms** you intend to solve it
% - MoSCoW analysis of the functional and non-functional features you intend to implement for your solution

% Note: you are **not** explaining the specifics of how; this will be done in the following chapters! 

% You are just saying what Must, Should,Could and Would be implemented to be able to solve the problem described above

\begin{itemize}
    \item \textbf{Must:}
    \begin{itemize}
        \item The system must be able to communicate with the \acs{HIS}, i.e. capable of processing incoming requests and transmit necessary data.
        \item All communications within the system must be secure and any sensitive data cannot be accessed by unauthorized third parties.
        \item The system must be able to function for long periods of time without continuous support or maintenance.
        \item The system must be non-invasive and intuitive.
    \end{itemize}
    \item \textbf{Should:}
    \begin{itemize}   
        \item The system should adopt international standards for exchanging information.
    \end{itemize}
    \item \textbf{Could:}
    \begin{itemize}   
        \item With long-term monitoring, large amounts of information could be gathered to create valuable datasets that can be used to improve patient monitoring.
    \end{itemize}
    \item \textbf{Would:}
    \begin{itemize}
        \item For long-term patient monitoring, the system would be capable of identifying anomalies / biomarkers in patients' biosignals.
    \end{itemize}
\end{itemize}

\section{Dissertation Structure}

This document is organized into different sections. The first chapter provides an introduction to the theme of the dissertation, discussing the context and motivation behind the work carried out. In the second chapter a brief overview into \acs{IoT} infrastructures and its healthcare applications is shown, along with a statement of the contributions of this work. The third chapter focuses on hardware analysis for one of the \acs{IoT} system components, the \textit{SmartBox}. The fourth chapter describes the service architecture within another system component, the \textit{Smart Gateway}, which is validated experimentally and analyzed in the fifth chapter.
Finally, in the sixth and final chapter, we reflect upon the work developed and discuss the completion of the objectives and on future work.