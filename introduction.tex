%INTRODUCTION: deve responder a "WHAT?" e a "WHY?"

% This report describes the work developed under the project "Wireless IoT Architecture for Smart Nodes deployed in Hospital Beds", which took place in the Institute of Systems and Robotics (ISR) in Coimbra in the past year. 

\section{Context}

% Due to all the technological and healthcare advances over the last years, we have observed a steady increase of life expectancy. A consequence of this fact is that 

% The steadily aging population [1] and related rise of chronical illnesses places significant strain on modern healthcare systems [2]. 

% E especialmente tendo em conta a situação pandémica que vivemos, observa-se uma necessidade clara de aliviar essa pressão. Felizmente, a Internet das Coisas é vista como um grande promessa para resolver esta questão.

% Podemos hoje encontrar iniciativas que caminham nesta direção como na automatização de certos processos (como a monitorização periódica dos pacientes), aliviando a alocação de recursos, e até na área de investigação para encontrar correlações entre determinados sinais biológicos e certas patologias (também designadas de biomarcadores), permitindo efetuar diagnósticos mais rápidos e com maior facilidade.

% Today, hospitalized patients need to be wired to various measurement instruments when continuous biomonitoring is required. This confines the patients to their beds, restricting their mobility, which can cause skin irritations and infections, aggravating their discomfort and deterioration of their health condition \cite{Darwish2011}. Moreover, the detachment of electrodes from the patient's body, provoked by patient's movements, is one of the main sources of false alarms. These require immediate attention from the hospital staff, contributing to their exhaustion and may ultimately result in the desensitization to the alarms, reducing their response time to real emergencies \cite{DursunErgezen2020}.


\todo[inline]{To-do:
Steady increase of population lifespan introduces many challenges to healthcare systems (more elderly people, chronic diseases become more common, thus greater pressure on these systems, bigger healthcare costs, ...);

What has digital health done to help this? concepts: IoT, digital health...}

\section{Objectives}
\todo[inline]{To-do: Discuss if this section should move to AFTER literature review}
% The main objective of this work is the development and validation of an \acs{IoT} architecture in the context of the \acs{WoW} project. Previous work by \acs{ISR} researchers has resulted in the development of wearable devices, designated ``biostickers''...
% The system should be non-invasive, reliable and satisfy the stringent security and privacy requirements of health information systems. 
% - Hardware evaluation for edge nodes which integrate electronic wireless patches that gather patient's physiological signals;
% - Integrating IoT system in an existing healthcare information system (Glintt GlobalCare software) through an FHIR API layer;
% - Evaluation of the performance of the proposed system through a testbed and a real healthcare scenario;
\section{Thesis Structure}

This document is organized into different sections. The first chapter provides an introduction to the theme of the dissertation, discussing the context and motivation behind the work developed. In the second chapter a brief overview into \acs{IoT} infrastructures and its healthcare applications is shown. 

%\todo[inline]{To-do: Apresentar e explicar a estrutura / organização da tese "dizer como vamos responder a "HOW?" na restante dissertação."}