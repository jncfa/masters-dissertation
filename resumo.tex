% Resumo em Português
\vspace{1cm}
\noindent

Nos últimos anos, em particular com o desenvolvimento da pandemia COVID-19, temos observado uma transformação digital continua dos cuidados de saúde. Existe uma tecnologia em particular que revela grande potencial para revolucionar o paradigma atual dos cuidados de saúde -- Internet das Coisas (\acs{IoT}).

Este trabalho consiste no desenvolvimento de uma infrastrutura \acs{IoT} que conecta sensores inteligentes vestíveis a um sistema de informação de saúde utilizado por hospitais nacionais. Em particular, uma avaliação de placas de hardware \acs{IoT} comuns é efetuada para determinar que placa será utilizada para a infraestrutura, além de uma análise da comunicação \acf{BLE}. No âmbito deste trabalho foi também proposta uma especificação para comunicação \acf{MQTT}, tal como o desenvolvimento de uma Interface de Programação de Aplicações (\acs{API}) usando o protocolo \acf{FHIR} para integrar a infraestrutura no \acs{HIS}, juntamente com uma arquitetura de servições para o servidor \textit{edge}, desenhado com segurança e privacidade em mente.

A plataforma desenvolvida é validada através de um ensaio em instalações hospitalares e testes em ambiente controlado, nos quais a infraestrutura mostrou uso eficiente de recursos, disponibilidade máxima e boa segurança, fornecendo uma base sólida para continuar o desenvolvimento e expandir as funcionalidades existêntes do sistema.


\paragraph{}\textbf{Palavras-chave:} IoT; Saúde; Sensores vestíveis; Cuidados de saúde remotos; MQTT; BLE; FHIR.