


% - Restate your research topic
% - Reiterate the Main Contribution of the Thesis
% - Summarize the Work Succinctly
% - Significance of Results
% - Consider the Larger Context/other Applications
% - Point to the Future: Next Steps

In this document a new \acs{IoT} architecture for a pervasive healthcare application has been proposed and implemented, as well as an extensive study of \acs{BLE} communication using \textit{Linux} and a hardware evaluation of different \acs{SBC}s. This work also evaluates the performance of two \acs{BLE} adapters, exposing issues with the \acs{BLE} stack implementation on \textit{Linux} and their significance. Additionally, a \acs{MQTT} specification using standardized data formats and security protocols, as well as an extensive service architecture within the \textit{Smart Gateway}, has been proposed, promoting security and interoperability in healthcare \acs{IT} systems. Unfortunately, due to external causes the performance could not be evaluated within a trial with clinical facilities, but valuable insight was obtained from it and the results in the controlled lab tests are very promising, demonstrating the current capabilities and potential of the proposed solution. 

\paragraph{} As a result of the developed work and its contributions to the \acs{WoW} project, an article \cite{Fama2021} has been co-written and submitted to the Internet of Things journal published by Elsevier, pending review at the time of writing. 

\section{Future Work}

To further improve the current work, some issues of the proposed solutions must be tackled. The proposed \acs{MQTT} specification does not implement any redundancy mechanisms exchanging data after interruptions in the communication, \textit{e.g.} network failures, which are required to make the system more robust. Additionally, as mentioned previously, the Data pre-processing service does not implement analytics to detect critical conditions. This would improve significantly the appeal of the system, as it would be capable of providing insight into the patients' conditions in real-time, \textit{e.g.} detecting heart attacks or fall events, that could be used to alert the healthcare providers much faster. Moreover, the \acs{FHIR} server currently does not support the creation or modification of subscriptions using the \acs{HTTP} \acs{API} as it was not considered for the first trial, but is necessary for the final stage of the project. 