


% - Restate your research topic
% - Reiterate the Main Contribution of the Thesis
% - Summarize the Work Succinctly
% - Significance of Results
% - Consider the Larger Context/other Applications
% - Point to the Future: Next Steps

In this dissertation, a new \acs{IoT} architecture for a pervasive healthcare application of long-term wireless biomonitoring of patients has been proposed and implemented. The work includes an extensive study of \acs{BLE} communication and a hardware evaluation of different \acs{SBC}s to support the decision of which \acs{IoT} hardware platforms would be used in the \acs{WoW} project. This work also evaluates the performance of two \acs{BLE} adapters, identifying issues with the \acs{BLE} stack implementation on \textit{Linux} and their significance to the community. Additionally, a \acs{MQTT} specification using standardized data formats and security protocols, as well as an extensive service architecture within the \textit{Smart Gateway}, has been proposed, promoting security and interoperability in healthcare \acs{IT} systems by {\color{blue} using} an \acs{API} {\color{blue} built with} the open standard \acs{FHIR} to integrate the data in an existing \acs{HIS}. 

\paragraph{} Due to external causes, the performance could not be fully evaluated within a trial in clinical facilities, but valuable insight was obtained from it. To overcome the weaknesses of the preliminary tests, results in a controlled lab scenario were extracted for further evaluation. These tests were very promising, showing how the system is more than capable of handling the communications bandwidths required for the project while remaining extremely secure and scalable, thus demonstrating the current capabilities and potential of the proposed solution. 

\paragraph{} As a result of the developed work and its contributions to the \acs{WoW} project, an article \cite{Fama2021} has been co-written and submitted to the Internet of Things journal published by Elsevier, pending review at the time of writing. 

\section{Future Work}

To further improve the current work, some open issues of the proposed solutions could be tackled. The proposed \acs{MQTT} specification does not implement any redundancy mechanisms exchanging data after interruptions in the communication, \textit{e.g.} network failures, which are required to make the system more robust. Additionally, as mentioned previously, the Data pre-processing service does not implement analytics to detect critical conditions. This would improve significantly the appeal of the system, as it would be capable of providing insight into the patients' conditions in real-time, \textit{e.g.} detecting the eminence of heart attacks or fall events, triggering an immediate alert to healthcare providers. Moreover, the \acs{FHIR} server currently does not support the creation or modification of subscriptions using the \acs{HTTP} \acs{API} as it was not considered for the first trial. However, this should be included in the planning for the final stage of the project. 