% Abstract in English

\vspace{1cm}
\noindent

In the past years, particularly with the COVID-19 pandemic, we have observed the continuous digital transformation of healthcare. But one technology in particular shows potential to revolutionize the current healthcare paradigm -- the  \acf{IoT}.
The current research gaps surrounding these types of systems lie on their lack of interoperability with other health information systems, and more robust security and privacy measures.

This work focuses on the development of a secure \acs{IoT} infrastructure which connects wireless, wearable, and intelligent sensors to a health information system used by multiple national hospitals. 
In particular, an extensive benchmarking of common \acs{IoT} hardware platforms to serve as \acs{IoT} acquisition nodes is performed, and an analysis of \acf{BLE} communication to evaluate the adequacy of the devices used to implement it. 
A specification for \acf{MQTT} communication is also proposed, as well as the development of a \acf{API} using \acf{FHIR}, an open standard widely used in health informatics for exchanging electronic health records, to integrate the \acs{IoT} infrastructure with the health information system, coupled with an efficient and scalable service architecture for the edge server, designed with security and privacy in mind. 

The work is validated both through hospital trials and controlled lab tests, in which the infrastructure shows efficient resource usage, maximum availability and security, providing solid foundations for future work to build upon. 
\paragraph{}\textbf{Keywords:} \acl{IoT}; \acs{FHIR}; \acs{MQTT}; \acl{BLE}; Healthcare; Wearable sensors.
